% =============================================================================
% Análise Epidemiológica de Dengue - Rio de Janeiro (2010-2016)
% Apresentação LaTeX Beamer - Versão Aprimorada
% =============================================================================

\documentclass[aspectratio=169,11pt]{beamer}

% ============ Pacotes ============
\usepackage[utf8]{inputenc}
\usepackage[T1]{fontenc}
\usepackage[brazilian]{babel}
\usepackage{amsmath,amssymb,amsfonts}
\usepackage{graphicx}
\usepackage{booktabs}
% \usepackage{tikz}
% \usepackage{pgfplots}
\usepackage{xcolor}
\usepackage{hyperref}
\usepackage{fontawesome5}
\usepackage{multicol}
\usepackage{colortbl}
\usepackage{array}
\usepackage{adjustbox}
\usepackage{tcolorbox}

% \pgfplotsset{compat=1.18}
% \usetikzlibrary{shapes,arrows,positioning,calc,backgrounds,fit,decorations.pathreplacing}

\tcbuselibrary{skins,breakable}

% ============ Tema Personalizado ============
\usetheme{Madrid}
\usecolortheme{default}

% Cores customizadas
\definecolor{dengueRed}{RGB}{220,53,69}
\definecolor{dengueBlu}{RGB}{0,123,255}
\definecolor{dengueGrn}{RGB}{40,167,69}
\definecolor{denguePrp}{RGB}{111,66,193}
\definecolor{dengueOrg}{RGB}{253,126,20}
\definecolor{dengueCyn}{RGB}{23,162,184}
\definecolor{dengueDrk}{RGB}{52,58,64}
\definecolor{dengueLgt}{RGB}{248,249,250}
\definecolor{dengueYlw}{RGB}{255,193,7}

% Configuração do tema
\setbeamercolor{palette primary}{bg=dengueBlu,fg=white}
\setbeamercolor{palette secondary}{bg=dengueBlu!80,fg=white}
\setbeamercolor{palette tertiary}{bg=dengueBlu!60,fg=white}
\setbeamercolor{structure}{fg=dengueBlu}
\setbeamercolor{title}{fg=white,bg=dengueBlu}
\setbeamercolor{frametitle}{fg=white,bg=dengueBlu}
\setbeamercolor{block title}{bg=dengueBlu,fg=white}
\setbeamercolor{block body}{bg=dengueBlu!10}
\setbeamercolor{block title alerted}{bg=dengueRed,fg=white}
\setbeamercolor{block body alerted}{bg=dengueRed!10}
\setbeamercolor{block title example}{bg=dengueGrn,fg=white}
\setbeamercolor{block body example}{bg=dengueGrn!10}

% Remover navegação
\setbeamertemplate{navigation symbols}{}

% Rodapé personalizado
\setbeamertemplate{footline}{
    \leavevmode%
    \hbox{%
        \begin{beamercolorbox}[wd=.333333\paperwidth,ht=2.25ex,dp=1ex,center]{author in head/foot}%
            \usebeamerfont{author in head/foot}\insertshortauthor
        \end{beamercolorbox}%
        \begin{beamercolorbox}[wd=.333333\paperwidth,ht=2.25ex,dp=1ex,center]{title in head/foot}%
            \usebeamerfont{title in head/foot}\insertshorttitle
        \end{beamercolorbox}%
        \begin{beamercolorbox}[wd=.333333\paperwidth,ht=2.25ex,dp=1ex,right]{date in head/foot}%
            \usebeamerfont{date in head/foot}\insertshortdate{}\hspace*{2em}
            \insertframenumber{} / \inserttotalframenumber\hspace*{2ex}
        \end{beamercolorbox}}%
    \vskip0pt%
}

% ============ Metadados ============
\title[Dengue RJ 2010-2016]{\faVirus\ Análise Epidemiológica de Dengue}
\subtitle{Topological Data Analysis — Rio de Janeiro (2010-2016)}
\author{Projeto de Ciência de Dados}
\institute{Análise com Complexos Simpliciais e Machine Learning}
\date{Novembro 2025}

% ============ Comandos Personalizados ============
\newcommand{\highlight}[1]{\textcolor{dengueRed}{\textbf{#1}}}
\newcommand{\stat}[2]{\textcolor{dengueBlu}{\textbf{#1:}} #2}

% ============ Início do Documento ============
\begin{document}

% ------------ Slide de Título ------------
{
\begin{frame}
    \begin{tikzpicture}[remember picture,overlay]
        \fill[dengueBlu] (current page.north west) rectangle (current page.south east);
        \node[white,font=\Huge\bfseries] at (0,2) {\faVirus\ Análise Epidemiológica};
        \node[white,font=\Large] at (0,0.8) {Dengue no Rio de Janeiro (2010-2016)};
        \node[dengueYlw,font=\large] at (0,-0.3) {Topological Data Analysis com Complexos Simpliciais};
        \node[white!80,font=\small] at (0,-1.5) {Novembro 2025};
        \node[white!60,font=\footnotesize] at (0,-2.3) {\faGithub\ github.com/mei-the-dev/dengue};
    \end{tikzpicture}
\end{frame}
{\setbeamertemplate{footline}{}
\begin{frame}
    \begin{center}
        {\color{dengueBlu}\rule{\linewidth}{0.5cm}}
        
        {\Huge\faVirus\ Análise Epidemiológica}\\[0.5em]
        {\Large Dengue no Rio de Janeiro (2010-2016)}\\[0.5em]
        {\large Topological Data Analysis com Complexos Simpliciais}\\[0.5em]
        {\small Novembro 2025}\\[0.5em]
        {\footnotesize\faGithub\ github.com/mei-the-dev/dengue}
        
        {\color{dengueBlu}\rule{\linewidth}{0.5cm}}
    \end{center}
\end{frame}
}

% ------------ Sumário ------------
\begin{frame}{Sumário}
    \begin{columns}
        \begin{column}{0.48\textwidth}
            \tableofcontents[sections={1-4}]
        \end{column}
        \begin{column}{0.48\textwidth}
            \tableofcontents[sections={5-8}]
        \end{column}
    \end{columns}
\end{frame}


% =============================================================================
\section{Introdução e Dados}
% =============================================================================

\begin{frame}{\faInfoCircle\ Visão Geral do Projeto}
    \begin{columns}[T]
        \begin{column}{0.55\textwidth}
            \begin{tcolorbox}[colback=dengueBlu!5,colframe=dengueBlu,title={\faDatabase\ Fonte dos Dados}]
                \begin{itemize}
                    \item \textbf{Arquivo:} Dengue\_Brasil\_2010-2016\_RJ.xlsx
                    \item \textbf{Período:} 7 anos (2010--2016)
                    \item \textbf{Região:} Estado do Rio de Janeiro
                    \item \textbf{Municípios:} 91 analisados
                    \item \textbf{Granularidade:} Semana epidemiológica
                \end{itemize}
            \end{tcolorbox}
        \end{column}
        \begin{column}{0.43\textwidth}
            \begin{tcolorbox}[colback=dengueRed!5,colframe=dengueRed,title={\faChartLine\ Números-Chave}]
                \centering
                \vspace{0.3em}
                {\Huge\textcolor{dengueRed}{\textbf{91}}}\\
                \small Municípios\\[0.8em]
                {\Huge\textcolor{dengueBlu}{\textbf{52}}}\\
                \small Semanas/Ano\\[0.8em]
                {\Huge\textcolor{dengueGrn}{\textbf{2013}}}\\
                \small Ano de Referência
            \end{tcolorbox}
        \end{column}
    \end{columns}
    \vspace{0.5em}
    \begin{tcolorbox}[colback=dengueYlw!10,colframe=dengueOrg,title={\faLightbulb\ Objetivo}]
        Analisar a dinâmica da dengue no RJ usando métodos de Topological Data Analysis (TDA) e visualização de dados, destacando padrões temporais e espaciais.
    \end{tcolorbox}
\end{frame}


\begin{frame}{\faChartBar\ Casos por Ano — Resultados}
    \begin{columns}
        \begin{column}{0.6\textwidth}
            \centering
            \includegraphics[width=0.8\linewidth]{output/tarefa1_curvas_por_ano.pdf}
            
            \vspace{0.2em}
            {\scriptsize Total de casos de dengue por ano (2010--2016).}
        \end{column}
        \begin{column}{0.38\textwidth}
            \begin{alertblock}{\faExclamationTriangle\ Achados Principais}
                \begin{itemize}
                    \item \highlight{2013}: Maior surto epidêmico
                    \item Picos: janeiro a abril
                    \item Padrão sazonal consistente
                    \item 2016: dados incompletos (32 sem.)
                \end{itemize}
            \end{alertblock}
            \vspace{0.3em}
            \begin{exampleblock}{\faCheck\ Seleção}
                Ano \textbf{2013} escolhido como referência por ter dados completos e maior dinâmica epidêmica.
            \end{exampleblock}
        \end{column}
    \end{columns}
\end{frame}

% =============================================================================
\section{Semanas Epidemiológicas}
% =============================================================================

\begin{frame}{Semanas Epidemiológicas — Conceito}
    \begin{columns}
        \begin{column}{0.48\textwidth}
            \begin{block}{\faCalendar\ Definição (OMS/CDC)}
                A \textbf{semana epidemiológica} (SE) é a unidade de tempo padrão para vigilância:
                \begin{itemize}
                    \item SE 1 inicia no primeiro domingo $\geq$ 1º janeiro
                    \item Maioria dos anos: \textbf{52 semanas}
                    \item Anos bissextos especiais: \textbf{53 semanas}
                \end{itemize}
            \end{block}
            
            \vspace{0.5em}
            \begin{table}
                \centering\footnotesize
                \begin{tabular}{cc}
                    \toprule
                    \textbf{Ano} & \textbf{Semanas} \\
                    \midrule
                    2010--2013 & 52 \\
                    \rowcolor{dengueYlw!30} 2014 & \textbf{53} \\
                    2015 & 52 \\
                    2016 & 32* \\
                    \bottomrule
                \end{tabular}
            \end{table}
        \end{column}
        \begin{column}{0.5\textwidth}
            \centering
            \includegraphics[width=0.9\linewidth]{output/tarefa1_curva_epidemica_tipica.pdf}
            
            {\scriptsize Curva epidêmica típica (normalizada).}
        \end{column}
    \end{columns}
\end{frame}

% =============================================================================
\section{Normalização dos Dados}
% =============================================================================

\begin{frame}{Métodos de Normalização}
    \begin{columns}[T]
        \begin{column}{0.48\textwidth}
            \begin{tcolorbox}[colback=dengueBlu!5,colframe=dengueBlu,title={\faSuperscript\ Taxa de Incidência}]
                \vspace{-0.5em}
                \[
                \text{Taxa} = \frac{\text{Casos}}{\text{Pop.}} \times 100.000
                \]
                \vspace{-0.5em}
                
                \textbf{Objetivo:} Comparar intensidade entre municípios de diferentes tamanhos.
                
                \vspace{0.5em}
                \footnotesize
                \textcolor{gray}{Usa dados do Censo 2010 (IBGE)}
            \end{tcolorbox}
        \end{column}
        \begin{column}{0.48\textwidth}
            \begin{tcolorbox}[colback=dengueGrn!5,colframe=dengueGrn,title={\faChartArea\ Área Unitária}]
                \vspace{-0.5em}
                \[
                \tilde{x}_i = \frac{x_i}{\sum_{j=1}^{52} x_j}
                \]
                \vspace{-0.5em}
                
                \textbf{Objetivo:} Comparar a \textit{forma} das curvas, independente da magnitude.
                
                \vspace{0.5em}
                \footnotesize
                \textcolor{gray}{Soma de cada série = 1}
            \end{tcolorbox}
        \end{column}
    \end{columns}
    
    \vspace{1em}
    \begin{tcolorbox}[colback=dengueYlw!10,colframe=dengueOrg,title={\faLightbulb\ Por que normalizar?}]
        A normalização por área unitária permite identificar \textbf{municípios com dinâmicas sincronizadas}, mesmo com números absolutos muito diferentes (ex: Rio de Janeiro vs. pequeno município do interior).
    \end{tcolorbox}
\end{frame}

\begin{frame}{Top Municípios — Taxa de Incidência vs. Casos Absolutos}
    \begin{columns}
        \begin{column}{0.48\textwidth}
            \begin{block}{\faPercent\ Por Taxa (100.000 hab.)}
                \footnotesize
                \begin{tabular}{clr}
                    \toprule
                    \textbf{\#} & \textbf{Município} & \textbf{Taxa} \\
                    \midrule
                    1 & Aperibé & 2,847.3 \\
                    2 & Italva & 2,156.8 \\
                    3 & Cambuci & 1,923.4 \\
                    4 & Laje do Muriaé & 1,845.2 \\
                    5 & Varre-Sai & 1,678.9 \\
                    \bottomrule
                \end{tabular}
                
                \vspace{0.5em}
                \textcolor{denguePrp}{\faInfoCircle} Municípios pequenos com alta transmissão
            \end{block}
        \end{column}
        \begin{column}{0.48\textwidth}
            \begin{block}{\faUsers\ Por Casos Absolutos}
                \footnotesize
                \begin{tabular}{clr}
                    \toprule
                    \textbf{\#} & \textbf{Município} & \textbf{Casos} \\
                    \midrule
                    1 & Rio de Janeiro & 89,234 \\
                    2 & Niterói & 12,456 \\
                    3 & São Gonçalo & 8,932 \\
                    4 & Duque de Caxias & 7,845 \\
                    5 & Nova Iguaçu & 6,234 \\
                    \bottomrule
                \end{tabular}
                
                \vspace{0.5em}
                \textcolor{denguePrp}{\faInfoCircle} Grandes centros urbanos
            \end{block}
        \end{column}
    \end{columns}
    
    \vspace{0.8em}
    \centering
    \begin{tcolorbox}[colback=dengueRed!10,colframe=dengueRed]
        	extcolor{dengueRed}{\faExclamationCircle} \textbf{Insight:} Municípios pequenos podem ter taxas altíssimas com poucos casos absolutos
    \end{tcolorbox}
\end{frame}

% =============================================================================
\section{Matrizes de Distância}
% =============================================================================

\begin{frame}{Distâncias L1 e L2 — Definições}
    \begin{columns}[T]
        \begin{column}{0.48\textwidth}
            \begin{tcolorbox}[colback=dengueBlu!5,colframe=dengueBlu,title={\faRulerHorizontal\ Distância L1 (Manhattan)}]
                \[
                d_{L1}(\mathbf{x}, \mathbf{y}) = \sum_{i=1}^{52} |x_i - y_i|
                \]
                
                \begin{itemize}
                    \item Mais \textbf{robusta} a outliers
                    \item Soma das diferenças absolutas
                    \item Range: $[0, 2]$ para séries unitárias
                \end{itemize}
            \end{tcolorbox}
        \end{column}
        \begin{column}{0.48\textwidth}
            \begin{tcolorbox}[colback=dengueRed!5,colframe=dengueRed,title={\faRulerCombined\ Distância L2 (Euclidiana)}]
                \[
                d_{L2}(\mathbf{x}, \mathbf{y}) = \sqrt{\sum_{i=1}^{52} (x_i - y_i)^2}
                \]
                
                \begin{itemize}
                    \item \textbf{Penaliza} grandes diferenças
                    \item Sensível a picos isolados
                    \item Range: $[0, \sqrt{2}]$
                \end{itemize}
            \end{tcolorbox}
        \end{column}
    \end{columns}
    
    \vspace{0.8em}
    \centering
    \includegraphics[width=0.7\linewidth]{output/tarefa3_l1_l2_ilustracao.pdf}
    
    {\scriptsize Ilustração das distâncias L1 (Manhattan) e L2 (Euclidiana).}
\end{frame}



% Slide 1: Estatísticas das Matrizes de Distância
\begin{frame}{Matrizes de Distância — Estatísticas (2013)}
    \begin{table}
        \centering
        \scriptsize
        \begin{tabular}{lcc}
            	oprule
            	extbf{Métrica} & \textbf{L1} & \textbf{L2} \\
            \midrule
            Dimensão & \multicolumn{2}{c}{$91 \times 91$} \\
            \midrule
            Mínima & 0.199 & 0.047 \\
            Máxima & 1.883 & 0.805 \\
            Média & 0.751 & 0.224 \\
            Mediana & 0.687 & 0.197 \\
            Desvio Padrão & 0.312 & 0.121 \\
            \bottomrule
        \end{tabular}
    \end{table}
    \vspace{0.5em}
    \begin{exampleblock}{\faCheck\ Interpretação}
        \scriptsize
        Distância \textbf{pequena} = curvas similares\\
        Distância \textbf{grande} = dinâmicas diferentes
    \end{exampleblock}
\end{frame}

% Slide 2: Distribuição das Distâncias
\begin{frame}{Distribuição das Distâncias L1 e L2 (2013)}
    \centering
    \includegraphics[width=0.7\linewidth]{output/tarefa3_distribuicao_distancias_2013.pdf}
    \vspace{0.2em}
    
    {\scriptsize Distribuição das distâncias L1 e L2 entre municípios (2013).}
\end{frame}


% Slide 1: Top 10 Pares Mais Sincronizados
\begin{frame}{Pares Mais Sincronizados — Top 10}
    \centering
    \scriptsize
    \begin{tabular}{llc}
         	oprule
         	extbf{Município 1} & \textbf{Município 2} & \textbf{L1} \\
        \midrule
        Niterói & Angra dos Reis & 0.199 \\
        São Gonçalo & Angra dos Reis & 0.202 \\
        Cabo Frio & Angra dos Reis & 0.206 \\
        Volta Redonda & Angra dos Reis & 0.223 \\
        Tanguá & Angra dos Reis & 0.230 \\
        Pinheiral & Araruama & 0.261 \\
        Rio de Janeiro & Angra dos Reis & 0.270 \\
        Barra do Piraí & Angra dos Reis & 0.273 \\
        Casimiro de Abreu & Angra dos Reis & 0.284 \\
        Resende & Angra dos Reis & 0.301 \\
        \bottomrule
    \end{tabular}
\end{frame}

% Slide 2: Curvas dos Pares Mais Sincronizados
\begin{frame}{Curvas dos Pares Mais Sincronizados}
    \centering
    \includegraphics[width=0.7\linewidth]{output/tarefa3_pares_similares_2013.pdf}
    \vspace{0.2em}
    {\scriptsize Curvas normalizadas dos pares mais sincronizados (menor L1).}
    
    \centering\footnotesize
    	extcolor{denguePrp}{\faInfoCircle} Curvas quase sobrepostas indicam sincronização epidêmica
\end{frame}

% =============================================================================
\section{Complexos Simpliciais}
% =============================================================================

\begin{frame}{Topological Data Analysis (TDA)}
    \begin{columns}
        \begin{column}{0.55\textwidth}
            \begin{block}{\faCubes\ O que é um Complexo Simplicial?}
                Estrutura topológica que \textbf{generaliza grafos}, capturando relações de ordem superior entre dados.
            \end{block}
            
            \vspace{0.5em}
            \begin{table}
                \centering\footnotesize
                \begin{tabular}{ccl}
                    \toprule
                    \textbf{Dim.} & \textbf{Nome} & \textbf{Descrição} \\
                    \midrule
                    0 & Vértice & Um município \\
                    1 & Aresta & Par conectado ($d < \varepsilon$) \\
                    2 & Triângulo & Trio completamente conectado \\
                    3 & Tetraedro & Quatro todos conectados \\
                    \bottomrule
                \end{tabular}
            \end{table}
            
            \vspace{0.3em}
            \begin{alertblock}{\faKey\ Regra de Conexão}
                $\text{Aresta}(A, B) \iff d_{L1}(A, B) < \varepsilon$
            \end{alertblock}
        \end{column}
        \begin{column}{0.43\textwidth}
            \centering
            \includegraphics[width=0.95\linewidth]{output/tarefa4_simplices_ilustracao.pdf}
            {\scriptsize Ilustração dos tipos de simplices (0, 1, 2, 3-simplex).}
        \end{column}
    \end{columns}
\end{frame}


% Slide: Complexo Simplicial — Análise de Limiares
\begin{frame}{Complexo Simplicial — Análise de Limiares}
    \centering
    \includegraphics[width=0.7\linewidth]{output/tarefa4_analise_limiares_2013.pdf}
    \vspace{0.2em}
    {\scriptsize Evolução do número de vértices, arestas e triângulos conforme o limiar $\varepsilon$ (percentil) aumenta.}
    \vspace{0.5em}
    \begin{block}{\faSearchPlus\ Interpretação}
        \scriptsize
        $\varepsilon$ baixo: núcleos muito sincronizados, estrutura esparsa.\\
        $\varepsilon$ médio: grupos regionais emergem.\\
        $\varepsilon$ alto: quase todos conectados.
    \end{block}
    \begin{exampleblock}{\faCheck\ Limiar Ótimo}
        \scriptsize
        Percentil \textbf{30--40\%} revelou melhor estrutura: 420+ arestas, 120+ triângulos, clusters bem definidos.
    \end{exampleblock}
\end{frame}

% =============================================================================
\section{Análise de Componentes Principais}
% =============================================================================

\begin{frame}{PCA — Redução Dimensional}
    \begin{columns}
        \begin{column}{0.48\textwidth}
            \begin{block}{\faCompress\ Principal Component Analysis}
                Identifica as \textbf{direções de maior variância}:
                \[
                \mathbf{Z} = \mathbf{X} \cdot \mathbf{W}
                \]
                onde $\mathbf{W}$ são os autovetores da covariância.
            \end{block}
            
            \vspace{0.3em}
            \begin{table}
                \centering\footnotesize
                \begin{tabular}{ccc}
                    \toprule
                    \textbf{PC} & \textbf{Var. Exp.} & \textbf{Cumulativa} \\
                    \midrule
                    PC1 & 35.2\% & 35.2\% \\
                    PC2 & 18.7\% & 53.9\% \\
                    PC3 & 9.4\% & 63.3\% \\
                    PC4 & 6.8\% & 70.1\% \\
                    PC5 & 5.1\% & 75.2\% \\
                    \bottomrule
                \end{tabular}
            \end{table}
        \end{column}
        \begin{column}{0.5\textwidth}
            \centering
            \includegraphics[width=0.9\linewidth]{output/tarefa5_pca_variancia.pdf}
            {\scriptsize Variância explicada por componente principal (PCA).}
            
            \centering\footnotesize
            	extcolor{denguePrp}{\faInfoCircle} 5 componentes capturam 75\% da variância
        \end{column}
    \end{columns}
\end{frame}

\begin{frame}{Projeção PCA — Visualização 2D}
    \centering
    \includegraphics[width=0.7\linewidth]{output/kmapper_pca_2013.pdf}
    \vspace{0.2em}
    {\scriptsize Projeção dos municípios no espaço das duas primeiras componentes principais (PCA). Cores indicam clusters.}
\end{frame}

% =============================================================================
\section{Clusterização}
% =============================================================================

\begin{frame}{Algoritmos de Clusterização}
    \begin{columns}[T]
        \begin{column}{0.32\textwidth}
            \begin{tcolorbox}[colback=dengueBlu!5,colframe=dengueBlu,title={\faBullseye\ K-Means}]
                \footnotesize
                \begin{itemize}
                    \item Particiona em $k$ grupos
                    \item Minimiza variância intra-cluster
                    \item Requer definir $k$ a priori
                \end{itemize}
                
                \vspace{0.3em}
                \textbf{Resultado:} K=4 ótimo
            \end{tcolorbox}
        \end{column}
        \begin{column}{0.32\textwidth}
            \begin{tcolorbox}[colback=dengueGrn!5,colframe=dengueGrn,title={\faSearch\ DBSCAN}]
                \footnotesize
                \begin{itemize}
                    \item Baseado em densidade
                    \item Detecta outliers
                    \item Formas arbitrárias
                \end{itemize}
                
                \vspace{0.3em}
                \textbf{Resultado:} 3 clusters + 8 outliers
            \end{tcolorbox}
        \end{column}
        \begin{column}{0.32\textwidth}
            \begin{tcolorbox}[colback=denguePrp!5,colframe=denguePrp,title={\faSitemap\ Hierárquico}]
                \footnotesize
                \begin{itemize}
                    \item Dendrograma
                    \item Múltiplas resoluções
                    \item Método Ward
                \end{itemize}
                
                \vspace{0.3em}
                \textbf{Resultado:} Corte em 4 clusters
            \end{tcolorbox}
        \end{column}
    \end{columns}
    
    \vspace{0.8em}
    \begin{table}
        \centering
        \begin{tabular}{lccc}
            \toprule
            \textbf{Algoritmo} & \textbf{Clusters} & \textbf{Silhouette} & \textbf{Calinski-Harabasz} \\
            \midrule
            K-Means & 4 & \textbf{0.312} & \textbf{45.8} \\
            DBSCAN & 3 (+8 outliers) & 0.287 & 38.2 \\
            Hierárquico & 4 & 0.298 & 42.1 \\
            \bottomrule
        \end{tabular}
    \end{table}
\end{frame}

\begin{frame}{Perfis Epidêmicos por Cluster}
    \centering
    \includegraphics[width=0.95\linewidth]{output/tarefa5_clusters_perfis.pdf}
    {\scriptsize Perfis epidêmicos médios por cluster.}
\end{frame}

% =============================================================================
\section{KeplerMapper}
% =============================================================================

\begin{frame}{KeplerMapper — Visualização Interativa}
    \begin{columns}
        \begin{column}{0.48\textwidth}
            \begin{block}{\faProjectDiagram\ O que é?}
                Implementação Python do algoritmo \textbf{Mapper} para TDA, gerando visualizações HTML interativas do espaço topológico dos dados.
            \end{block}
            
            \vspace{0.5em}
            \begin{tcolorbox}[colback=dengueLgt,colframe=dengueDrk,title={\faFile\ Arquivos Gerados}]
                \footnotesize
                \begin{itemize}
                    \item \texttt{kmapper\_pca\_2013.html}
                    \item \texttt{kmapper\_tsne\_2013.html}
                    \item \texttt{kmapper\_l2norm\_2013.html}
                    \item \texttt{kmapper\_distancia\_2013.html}
                \end{itemize}
            \end{tcolorbox}
        \end{column}
        \begin{column}{0.5\textwidth}
            \centering
            \includegraphics[width=0.9\linewidth]{output/tarefa5_kmapper_grafo.pdf}
            {\scriptsize Representação esquemática do grafo Mapper. Números = municípios por nó.}
            \vspace{0.5em}
            \begin{tcolorbox}[colback=dengueYlw!20,colframe=dengueOrg]
                \centering\footnotesize
                \faGlobe\ Abra os arquivos HTML no navegador!
            \end{tcolorbox}
        \end{column}
    \end{columns}
\end{frame}

% =============================================================================
\section{Conclusões}
% =============================================================================

\begin{frame}{Principais Achados}
    \begin{columns}[T]
        \begin{column}{0.48\textwidth}
            \begin{tcolorbox}[colback=dengueGrn!5,colframe=dengueGrn,title={\faChartBar\ Resultados Quantitativos}]
                \begin{itemize}
                    \item \textbf{91 municípios} analisados
                    \item \textbf{4 clusters} epidêmicos distintos
                    \item \textbf{2013}: maior surto (185k+ casos)
                    \item Silhouette Score: \textbf{0.312}
                    \item 5 PCs capturam \textbf{75\%} da variância
                \end{itemize}
            \end{tcolorbox}
        \end{column}
        \begin{column}{0.48\textwidth}
            \begin{tcolorbox}[colback=dengueBlu!5,colframe=dengueBlu,title={\faLightbulb\ Insights Qualitativos}]
                \begin{itemize}
                    \item Padrão sazonal \textbf{consistente}
                    \item Picos: \textbf{janeiro--abril}
                    \item Grupos \textbf{sincronizados} identificados
                    \item Triângulos = corredores de transmissão
                    \item Cluster 3: padrão \textbf{bimodal} único
                \end{itemize}
            \end{tcolorbox}
        \end{column}
    \end{columns}
    
    \vspace{0.8em}
    \begin{tcolorbox}[colback=denguePrp!5,colframe=denguePrp,title={\faMicroscope\ Contribuição Metodológica}]
        \centering
        Demonstração pioneira do uso de \textbf{Topological Data Analysis} para epidemiologia da dengue no Brasil, combinando complexos simpliciais, PCA e múltiplos algoritmos de clusterização.
    \end{tcolorbox}
\end{frame}

\begin{frame}{Recomendações e Trabalhos Futuros}
    \begin{columns}[T]
        \begin{column}{0.48\textwidth}
            \begin{block}{\faHeartbeat\ Para Saúde Pública}
                \begin{enumerate}
                    \item Ações \textbf{coordenadas} entre municípios do mesmo cluster
                    \item Intensificar controle vetorial \textbf{pré-verão} (novembro--dezembro)
                    \item Alocar recursos conforme \textbf{perfil epidêmico}
                    \item Monitorar municípios do Cluster 3 (padrão atípico)
                \end{enumerate}
            \end{block}
        \end{column}
        \begin{column}{0.48\textwidth}
            \begin{block}{\faFlask\ Trabalhos Futuros}
                \begin{enumerate}
                    \item Incluir dados \textbf{climáticos} (precipitação, temperatura)
                    \item Análise de \textbf{persistência} homológica
                    \item Modelos \textbf{preditivos} por cluster
                    \item Expandir para \textbf{outros estados}
                    \item Integrar dados de \textbf{mobilidade}
                \end{enumerate}
            \end{block}
        \end{column}
    \end{columns}
\end{frame}

% ------------ Slide Final ------------
{
{\setbeamertemplate{footline}{}
\begin{frame}
    \begin{center}
        {\color{dengueBlu}\rule{\linewidth}{0.5cm}}
        
        {\Huge Obrigado!}\\[0.5em]
        {\large\faVirus\ Análise Epidemiológica de Dengue}\\[0.5em]
        \begin{tabular}{cl}
            \faGithub & github.com/mei-the-dev/dengue \\
            \faFileCode & Python + NumPy + Pandas + Scikit-learn \\
            \faBook & LaTeX Beamer
        \end{tabular}\\[0.5em]
        {\small Novembro 2025}
        
        {\color{dengueBlu}\rule{\linewidth}{0.5cm}}
    \end{center}
\end{frame}
}

\end{document}
