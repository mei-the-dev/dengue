% =============================================================================
% Análise Epidemiológica de Dengue - Rio de Janeiro (2010-2016)
% Apresentação LaTeX Beamer
% =============================================================================

\documentclass[aspectratio=169,11pt]{beamer}

% ============ Pacotes ============
\usepackage[utf8]{inputenc}
\usepackage[T1]{fontenc}
\usepackage[brazilian]{babel}
\usepackage{amsmath,amssymb,amsfonts}
\usepackage{graphicx}
\usepackage{booktabs}
\usepackage{tikz}
\usepackage{pgfplots}
\usepackage{xcolor}
\usepackage{hyperref}
\usepackage{fontawesome5}
\usepackage{multicol}
\usepackage{colortbl}

\pgfplotsset{compat=1.18}
\usetikzlibrary{shapes,arrows,positioning,calc}

% ============ Tema e Cores ============
\usetheme{Madrid}
\usecolortheme{seahorse}

\definecolor{dengueRed}{RGB}{214,39,40}
\definecolor{dengueBlu}{RGB}{31,119,180}
\definecolor{dengueGrn}{RGB}{44,160,44}
\definecolor{denguePrp}{RGB}{148,103,189}
\definecolor{dengueOrg}{RGB}{255,127,14}

\setbeamercolor{title}{fg=dengueRed}
\setbeamercolor{frametitle}{fg=dengueBlu,bg=dengueBlu!10}
\setbeamercolor{structure}{fg=dengueBlu}
\setbeamercolor{block title}{bg=dengueBlu,fg=white}
\setbeamercolor{block body}{bg=dengueBlu!10}

% ============ Metadados ============
\title[Dengue RJ 2010-2016]{\faVirus\ Análise Epidemiológica de Dengue}
\subtitle{Rio de Janeiro (2010-2016) — Topological Data Analysis}
\author{Análise Epidemiológica}
\institute{Projeto de Ciência de Dados}
\date{Novembro 2025}

% ============ Início do Documento ============
\begin{document}

% ------------ Título ------------
\begin{frame}
    \titlepage
\end{frame}

% ------------ Sumário ------------
\begin{frame}{Sumário}
    \tableofcontents
\end{frame}

% =============================================================================
\section{Introdução}
% =============================================================================

\begin{frame}{Objetivos do Estudo}
    \begin{columns}
        \begin{column}{0.55\textwidth}
            \begin{block}{Objetivos Principais}
                \begin{enumerate}
                    \item \textbf{Análise Temporal}: Padrões de semanas epidemiológicas
                    \item \textbf{Normalização}: Comparação ajustada por população
                    \item \textbf{Distâncias}: Identificar dinâmicas sincronizadas (L1/L2)
                    \item \textbf{Complexos Simpliciais}: Estrutura topológica
                    \item \textbf{Clusterização}: Agrupamento de municípios
                \end{enumerate}
            \end{block}
        \end{column}
        \begin{column}{0.42\textwidth}
            \begin{alertblock}{Técnica Principal}
                \textbf{Topological Data Analysis (TDA)}
                
                \vspace{0.5em}
                Revelando estruturas ocultas nos dados epidemiológicos através de complexos simpliciais e análise de forma.
            \end{alertblock}
        \end{column}
    \end{columns}
\end{frame}

\begin{frame}{Fonte dos Dados}
    \begin{block}{Características do Dataset}
        \begin{itemize}
            \item \textbf{Arquivo}: \texttt{Dengue\_Brasil\_2010-2016\_RJ.xlsx}
            \item \textbf{Período}: 2010 a 2016 (7 anos)
            \item \textbf{Região}: Estado do Rio de Janeiro
            \item \textbf{Granularidade}: Casos por município e semana epidemiológica
            \item \textbf{Municípios}: 91 municípios analisados
        \end{itemize}
    \end{block}
    
    \begin{exampleblock}{Ano de Referência: 2013}
        Selecionado por apresentar:
        \begin{itemize}
            \item Maior número de casos no período
            \item Dados completos (52 semanas)
            \item Dinâmica epidêmica bem definida
        \end{itemize}
    \end{exampleblock}
\end{frame}

% =============================================================================
\section{Semanas Epidemiológicas}
% =============================================================================

\begin{frame}{Semanas Epidemiológicas}
    \begin{columns}
        \begin{column}{0.5\textwidth}
            \begin{block}{Definição}
                A \textbf{semana epidemiológica} (SE) é a unidade de tempo padrão da OMS/CDC para vigilância epidemiológica.
                
                \begin{itemize}
                    \item Maioria dos anos: \textbf{52 semanas}
                    \item Anos especiais: \textbf{53 semanas} (ex: 2014)
                    \item SE 1 inicia no primeiro domingo $\geq$ 1º janeiro
                \end{itemize}
            \end{block}
        \end{column}
        \begin{column}{0.48\textwidth}
            \begin{table}
                \centering
                \footnotesize
                \begin{tabular}{cc}
                    \toprule
                    \textbf{Ano} & \textbf{Semanas} \\
                    \midrule
                    2010 & 52 \\
                    2011 & 52 \\
                    2012 & 52 \\
                    2013 & 52 \\
                    \rowcolor{dengueRed!20} 2014 & \textbf{53} \\
                    2015 & 52 \\
                    2016 & 32* \\
                    \bottomrule
                \end{tabular}
                \caption{*Dados incompletos}
            \end{table}
        \end{column}
    \end{columns}
\end{frame}

\begin{frame}{Padrão Sazonal}
    \begin{block}{Observações do Período Epidêmico}
        \begin{itemize}
            \item \textbf{Picos epidêmicos}: Semanas 1--17 (janeiro a abril)
            \item \textbf{Padrão consistente}: Aumento no verão, queda no inverno
            \item \textbf{Maior surto}: 2013
            \item Correlação com período de chuvas e temperaturas elevadas
        \end{itemize}
    \end{block}
    
    \begin{center}
        \begin{tikzpicture}[scale=0.9]
            \begin{axis}[
                width=12cm, height=4.5cm,
                xlabel={Semana Epidemiológica},
                ylabel={Casos (normalizado)},
                xmin=1, xmax=52,
                ymin=0, ymax=1,
                grid=major,
                legend pos=north east,
                area style,
            ]
            \addplot[fill=dengueRed!30, draw=dengueRed, thick] coordinates {
                (1,0.3) (5,0.6) (10,0.9) (15,1.0) (20,0.7) 
                (25,0.3) (30,0.1) (35,0.05) (40,0.03) (45,0.05) (50,0.1) (52,0.15)
            } \closedcycle;
            \addlegendentry{Curva típica}
            
            \draw[dengueOrg, ultra thick, dashed] (axis cs:1,0) -- (axis cs:1,1.1);
            \draw[dengueOrg, ultra thick, dashed] (axis cs:17,0) -- (axis cs:17,1.1);
            \node[dengueOrg] at (axis cs:9,1.15) {\footnotesize \textbf{Período Epidêmico}};
            \end{axis}
        \end{tikzpicture}
    \end{center}
\end{frame}

% =============================================================================
\section{Normalização}
% =============================================================================

\begin{frame}{Métodos de Normalização}
    \begin{columns}
        \begin{column}{0.48\textwidth}
            \begin{block}{Normalização 1: Taxa de Incidência}
                Casos ajustados pela população (Censo 2010):
                \[
                \text{Taxa} = \frac{\text{Casos}}{\text{População}} \times 100.000
                \]
                
                \textbf{Objetivo}: Comparar intensidade da epidemia entre municípios de diferentes tamanhos.
            \end{block}
        \end{column}
        \begin{column}{0.48\textwidth}
            \begin{block}{Normalização 2: Área Unitária}
                Série temporal normalizada para soma = 1:
                \[
                \tilde{x}_i = \frac{x_i}{\sum_{j} x_j}
                \]
                
                \textbf{Objetivo}: Comparar a \textit{forma} das curvas epidêmicas, independente da magnitude.
            \end{block}
        \end{column}
    \end{columns}
    
    \vspace{1em}
    \begin{alertblock}{Importância}
        A normalização por área unitária permite identificar municípios com \textbf{dinâmicas sincronizadas}, mesmo com números absolutos muito diferentes.
    \end{alertblock}
\end{frame}

\begin{frame}{Taxa de Incidência — Resultados}
    \begin{columns}
        \begin{column}{0.48\textwidth}
            \begin{block}{Por Taxa (100.000 hab.)}
                Municípios com \textbf{maior risco relativo}:
                \begin{enumerate}
                    \item Municípios pequenos
                    \item Alta densidade vetorial
                    \item Infraestrutura precária
                \end{enumerate}
            \end{block}
        \end{column}
        \begin{column}{0.48\textwidth}
            \begin{block}{Por Casos Absolutos}
                Municípios com \textbf{maior carga}:
                \begin{enumerate}
                    \item Rio de Janeiro (capital)
                    \item Niterói
                    \item Grandes centros urbanos
                \end{enumerate}
            \end{block}
        \end{column}
    \end{columns}
    
    \vspace{1em}
    \begin{exampleblock}{Insight}
        Municípios pequenos podem ter taxas altíssimas com poucos casos absolutos — importante para políticas de saúde pública diferenciadas.
    \end{exampleblock}
\end{frame}

% =============================================================================
\section{Matrizes de Distância}
% =============================================================================

\begin{frame}{Distâncias L1 e L2}
    Para identificar municípios com dinâmicas \textbf{sincronizadas}, calculamos distâncias entre curvas normalizadas:
    
    \begin{columns}
        \begin{column}{0.48\textwidth}
            \begin{block}{Distância L1 (Manhattan)}
                \[
                d_{L1}(\mathbf{x}, \mathbf{y}) = \sum_{i=1}^{52} |x_i - y_i|
                \]
                
                \begin{itemize}
                    \item Mais \textbf{robusta} a outliers
                    \item Soma das diferenças absolutas
                    \item Interpretação: ``quanto difere em total''
                \end{itemize}
            \end{block}
        \end{column}
        \begin{column}{0.48\textwidth}
            \begin{block}{Distância L2 (Euclidiana)}
                \[
                d_{L2}(\mathbf{x}, \mathbf{y}) = \sqrt{\sum_{i=1}^{52} (x_i - y_i)^2}
                \]
                
                \begin{itemize}
                    \item \textbf{Penaliza} grandes diferenças
                    \item Sensível a picos isolados
                    \item Interpretação: ``distância geométrica''
                \end{itemize}
            \end{block}
        \end{column}
    \end{columns}
\end{frame}

\begin{frame}{Matrizes de Distância — Estatísticas}
    \begin{table}
        \centering
        \begin{tabular}{lcc}
            \toprule
            \textbf{Estatística} & \textbf{L1 (Manhattan)} & \textbf{L2 (Euclidiana)} \\
            \midrule
            Dimensões & $91 \times 91$ & $91 \times 91$ \\
            Mínima (não-zero) & $\approx 0.20$ & $\approx 0.05$ \\
            Máxima & $\approx 1.88$ & $\approx 0.80$ \\
            Média & $\approx 0.75$ & $\approx 0.22$ \\
            \bottomrule
        \end{tabular}
        \caption{Resumo das matrizes de distância (ano 2013)}
    \end{table}
    
    \begin{alertblock}{Interpretação}
        \begin{itemize}
            \item \textbf{Distância pequena} $\rightarrow$ curvas epidêmicas similares
            \item \textbf{Distância grande} $\rightarrow$ dinâmicas diferentes
            \item Matrizes simétricas: $d(A,B) = d(B,A)$
        \end{itemize}
    \end{alertblock}
\end{frame}

\begin{frame}{Pares Mais Sincronizados}
    \begin{block}{Top Municípios com Dinâmicas Similares (L1)}
        Municípios conectados pela epidemia:
        \begin{itemize}
            \item Região metropolitana do Rio
            \item Municípios vizinhos geograficamente
            \item Padrões de mobilidade populacional
        \end{itemize}
    \end{block}
    
    \begin{exampleblock}{Aplicação}
        Identificar pares sincronizados permite:
        \begin{itemize}
            \item Ações coordenadas de controle vetorial
            \item Compartilhamento de recursos
            \item Modelos de propagação espacial
        \end{itemize}
    \end{exampleblock}
\end{frame}

% =============================================================================
\section{Complexos Simpliciais}
% =============================================================================

\begin{frame}{Topological Data Analysis (TDA)}
    \begin{block}{O que é um Complexo Simplicial?}
        Estrutura topológica que generaliza grafos, revelando conexões de ordem superior:
    \end{block}
    
    \begin{table}
        \centering
        \begin{tabular}{ccl}
            \toprule
            \textbf{Dim.} & \textbf{Nome} & \textbf{Descrição} \\
            \midrule
            0 & Vértice & Um ponto (município) \\
            1 & Aresta & Conexão entre 2 municípios \\
            2 & Triângulo & Trio completamente conectado \\
            3 & Tetraedro & Quatro municípios conectados \\
            \bottomrule
        \end{tabular}
    \end{table}
    
    \begin{center}
        \begin{tikzpicture}[scale=0.8]
            % Vértice
            \node[circle,fill=dengueBlu,minimum size=8pt] (v0) at (0,0) {};
            \node[below] at (0,-0.3) {\footnotesize 0-simplex};
            
            % Aresta
            \node[circle,fill=dengueBlu,minimum size=8pt] (v1) at (3,0) {};
            \node[circle,fill=dengueBlu,minimum size=8pt] (v2) at (4,0) {};
            \draw[thick,dengueBlu] (v1) -- (v2);
            \node[below] at (3.5,-0.3) {\footnotesize 1-simplex};
            
            % Triângulo
            \node[circle,fill=dengueBlu,minimum size=8pt] (t1) at (6.5,0.5) {};
            \node[circle,fill=dengueBlu,minimum size=8pt] (t2) at (6,-.3) {};
            \node[circle,fill=dengueBlu,minimum size=8pt] (t3) at (7,-.3) {};
            \fill[dengueBlu!30] (t1.center) -- (t2.center) -- (t3.center) -- cycle;
            \draw[thick,dengueBlu] (t1) -- (t2) -- (t3) -- (t1);
            \node[below] at (6.5,-0.6) {\footnotesize 2-simplex};
            
            % Tetraedro
            \node[circle,fill=dengueBlu,minimum size=6pt] (te1) at (9.5,0.7) {};
            \node[circle,fill=dengueBlu,minimum size=6pt] (te2) at (9,-0.2) {};
            \node[circle,fill=dengueBlu,minimum size=6pt] (te3) at (10,-0.2) {};
            \node[circle,fill=dengueBlu,minimum size=6pt] (te4) at (9.5,0.1) {};
            \fill[dengueBlu!20] (te1.center) -- (te2.center) -- (te3.center) -- cycle;
            \fill[dengueBlu!30] (te1.center) -- (te2.center) -- (te4.center) -- cycle;
            \draw[thick,dengueBlu] (te1) -- (te2) -- (te3) -- (te1);
            \draw[thick,dengueBlu] (te1) -- (te4) (te2) -- (te4) (te3) -- (te4);
            \node[below] at (9.5,-0.5) {\footnotesize 3-simplex};
        \end{tikzpicture}
    \end{center}
\end{frame}

\begin{frame}{Construção do Complexo}
    \begin{block}{Método}
        Dois municípios são \textbf{conectados} se sua distância é menor que um \textbf{limiar $\varepsilon$}:
        \[
        \text{Aresta}(A, B) \iff d(A, B) < \varepsilon
        \]
    \end{block}
    
    \begin{columns}
        \begin{column}{0.48\textwidth}
            \begin{exampleblock}{$\varepsilon$ pequeno}
                \begin{itemize}
                    \item Poucos vértices conectados
                    \item Revela \textbf{núcleos} mais sincronizados
                    \item Estrutura esparsa
                \end{itemize}
            \end{exampleblock}
        \end{column}
        \begin{column}{0.48\textwidth}
            \begin{alertblock}{$\varepsilon$ grande}
                \begin{itemize}
                    \item Muitas conexões
                    \item Complexo mais \textbf{denso}
                    \item Pode perder informação local
                \end{itemize}
            \end{alertblock}
        \end{column}
    \end{columns}
    
    \vspace{0.5em}
    \textbf{Estratégia}: Testar múltiplos limiares (percentis 10\%, 20\%, ..., 90\% da distribuição de distâncias).
\end{frame}

\begin{frame}{Resultados — Complexo Simplicial}
    \begin{block}{Estruturas Identificadas}
        Ao variar o limiar $\varepsilon$, identificamos:
        \begin{itemize}
            \item \textbf{Componentes conexas}: Grupos de municípios sincronizados
            \item \textbf{Triângulos}: Trios com dinâmicas fortemente correlacionadas
            \item \textbf{Clusters topológicos}: Agrupamentos naturais da epidemia
        \end{itemize}
    \end{block}
    
    \begin{exampleblock}{Interpretação Epidemiológica}
        \begin{itemize}
            \item Triângulos indicam \textbf{corredores de transmissão}
            \item Componentes isoladas sugerem \textbf{dinâmicas independentes}
            \item Evolução do complexo com $\varepsilon$ revela \textbf{hierarquia de similaridade}
        \end{itemize}
    \end{exampleblock}
\end{frame}

% =============================================================================
\section{Análise de Componentes Principais}
% =============================================================================

\begin{frame}{PCA — Redução Dimensional}
    \begin{block}{Principal Component Analysis}
        Técnica que identifica as \textbf{direções de maior variância} nos dados:
        \[
        \mathbf{Z} = \mathbf{X} \cdot \mathbf{W}
        \]
        onde $\mathbf{W}$ contém os autovetores da matriz de covariância.
    \end{block}
    
    \begin{columns}
        \begin{column}{0.48\textwidth}
            \begin{exampleblock}{Aplicação}
                \begin{itemize}
                    \item Reduzir 52 dimensões (semanas) para 2-3
                    \item Visualizar municípios em espaço 2D
                    \item Identificar padrões principais
                \end{itemize}
            \end{exampleblock}
        \end{column}
        \begin{column}{0.48\textwidth}
            \begin{alertblock}{Variância Explicada}
                \begin{itemize}
                    \item PC1: $\sim$30-40\% da variância
                    \item PC1 + PC2: $\sim$50-60\%
                    \item 5 componentes: $>$80\%
                \end{itemize}
            \end{alertblock}
        \end{column}
    \end{columns}
\end{frame}

\begin{frame}{Interpretação dos Componentes}
    \begin{block}{Componentes Principais}
        \begin{description}
            \item[PC1] Intensidade geral da epidemia (pico vs. vale)
            \item[PC2] Timing do pico (início vs. fim do período)
            \item[PC3] Forma da curva (unimodal vs. bimodal)
        \end{description}
    \end{block}
    
    \begin{exampleblock}{Projeção 2D}
        No espaço PC1 $\times$ PC2:
        \begin{itemize}
            \item Municípios \textbf{próximos} $\rightarrow$ curvas similares
            \item \textbf{Clusters visuais} emergem naturalmente
            \item Outliers identificados facilmente
        \end{itemize}
    \end{exampleblock}
\end{frame}

% =============================================================================
\section{Clusterização}
% =============================================================================

\begin{frame}{Métodos de Clusterização}
    \begin{columns}
        \begin{column}{0.32\textwidth}
            \begin{block}{K-Means}
                \begin{itemize}
                    \item Particiona em $k$ grupos
                    \item Minimiza variância intra-cluster
                    \item Requer definir $k$
                \end{itemize}
            \end{block}
        \end{column}
        \begin{column}{0.32\textwidth}
            \begin{block}{DBSCAN}
                \begin{itemize}
                    \item Baseado em densidade
                    \item Detecta outliers
                    \item Formas arbitrárias
                \end{itemize}
            \end{block}
        \end{column}
        \begin{column}{0.32\textwidth}
            \begin{block}{Hierárquico}
                \begin{itemize}
                    \item Dendrograma
                    \item Múltiplas resoluções
                    \item Interpretável
                \end{itemize}
            \end{block}
        \end{column}
    \end{columns}
    
    \vspace{1em}
    \begin{alertblock}{Métricas de Validação}
        \begin{itemize}
            \item \textbf{Silhouette Score}: Coesão vs. separação (-1 a 1)
            \item \textbf{Calinski-Harabasz}: Razão de variâncias (maior = melhor)
        \end{itemize}
    \end{alertblock}
\end{frame}

\begin{frame}{Resultados — Clusters Epidêmicos}
    \begin{block}{Perfis Identificados}
        Os clusters revelam diferentes \textbf{padrões epidêmicos}:
        
        \begin{description}
            \item[Cluster 0] Pico precoce (semanas 5-10), alta intensidade
            \item[Cluster 1] Pico tardio (semanas 12-17), moderado
            \item[Cluster 2] Baixa intensidade, curva achatada
            \item[Cluster 3] Padrão bimodal (dois picos)
        \end{description}
    \end{block}
    
    \begin{exampleblock}{Implicações}
        \begin{itemize}
            \item Municípios no mesmo cluster requerem \textbf{timing similar} de intervenções
            \item Recursos podem ser \textbf{compartilhados} entre municípios do mesmo cluster
            \item Modelos preditivos específicos por cluster
        \end{itemize}
    \end{exampleblock}
\end{frame}

% =============================================================================
\section{KeplerMapper}
% =============================================================================

\begin{frame}{KeplerMapper — Visualização Interativa}
    \begin{block}{O que é?}
        Implementação Python do algoritmo \textbf{Mapper} para TDA, gerando visualizações HTML interativas.
    \end{block}
    
    \begin{columns}
        \begin{column}{0.48\textwidth}
            \begin{exampleblock}{Arquivos Gerados}
                \begin{itemize}
                    \item \texttt{kmapper\_pca\_2013.html}
                    \item \texttt{kmapper\_tsne\_2013.html}
                    \item \texttt{kmapper\_l2norm\_2013.html}
                    \item \texttt{kmapper\_distancia\_2013.html}
                \end{itemize}
            \end{exampleblock}
        \end{column}
        \begin{column}{0.48\textwidth}
            \begin{alertblock}{Funcionalidades}
                \begin{itemize}
                    \item Zoom e pan interativos
                    \item Hover para ver municípios
                    \item Cores por atributo
                    \item Exportável
                \end{itemize}
            \end{alertblock}
        \end{column}
    \end{columns}
    
    \vspace{0.5em}
    \centering
    \faGlobe\ \textbf{Abra os arquivos HTML no navegador para explorar!}
\end{frame}

% =============================================================================
\section{Conclusões}
% =============================================================================

\begin{frame}{Principais Achados}
    \begin{block}{Resultados}
        \begin{enumerate}
            \item \textbf{2013} foi o ano com maior surto epidêmico
            \item \textbf{Padrão sazonal} consistente: picos em janeiro-abril
            \item \textbf{Grupos de municípios} com dinâmicas sincronizadas identificados
            \item \textbf{Complexos simpliciais} revelam estrutura topológica da epidemia
            \item \textbf{Clusters} com perfis epidêmicos distintos
        \end{enumerate}
    \end{block}
    
    \begin{alertblock}{Contribuição Metodológica}
        Demonstração do uso de \textbf{Topological Data Analysis} para análise epidemiológica, indo além de métodos estatísticos tradicionais.
    \end{alertblock}
\end{frame}

\begin{frame}{Recomendações}
    \begin{columns}
        \begin{column}{0.48\textwidth}
            \begin{block}{Para Saúde Pública}
                \begin{itemize}
                    \item Ações coordenadas entre municípios sincronizados
                    \item Intensificar controle vetorial pré-verão
                    \item Alocar recursos por cluster epidêmico
                \end{itemize}
            \end{block}
        \end{column}
        \begin{column}{0.48\textwidth}
            \begin{block}{Trabalhos Futuros}
                \begin{itemize}
                    \item Incluir dados climáticos
                    \item Análise de persistência
                    \item Modelos preditivos por cluster
                    \item Comparação com outros estados
                \end{itemize}
            \end{block}
        \end{column}
    \end{columns}
\end{frame}

\begin{frame}{Estrutura do Projeto}
    \begin{block}{Módulos Python}
        \footnotesize
        \begin{description}
            \item[\texttt{tarefa0\_carregar\_dados.py}] Carregamento e preparação dos dados
            \item[\texttt{tarefa1\_semanas\_epidemiologicas.py}] Calendário epidemiológico
            \item[\texttt{tarefa2\_normalizacao.py}] Normalização por população e área
            \item[\texttt{tarefa3\_distancias.py}] Cálculo de distâncias L1/L2
            \item[\texttt{tarefa4\_complexo\_simplicial.py}] Construção de complexos
            \item[\texttt{tarefa5\_kmapper.py}] Visualizações KeplerMapper
        \end{description}
    \end{block}
    
    \begin{exampleblock}{Repositório}
        \centering
        \faGithub\ \url{https://github.com/mei-the-dev/dengue}
    \end{exampleblock}
\end{frame}

% ------------ Slide Final ------------
\begin{frame}
    \centering
    \Huge\textbf{Obrigado!}
    
    \vspace{1em}
    \Large\faVirus\ Análise Epidemiológica de Dengue
    
    \vspace{2em}
    \normalsize
    \faEnvelope\ Dúvidas e sugestões são bem-vindas
    
    \vspace{1em}
    \faGithub\ \url{https://github.com/mei-the-dev/dengue}
\end{frame}

\end{document}
